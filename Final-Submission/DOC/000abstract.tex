We are living in an information-explosion era where 2.5 Quintillion bytes of data are generated every day. It's an interesting and challenging work to dig deep into these data and mine "gold" out of them to present and take advantage of their intrinsic nature. Moreover, data are usually associated with multiple dimensions (e.g: a person's statistics may be presented with gender, nationality, height, weight, age, etc.). 

Tensors, or multi-dimensional data, are naturally expressed and stored as a table in relational databases with columns representing the value of each dimension. However, it's not always feasible to manipulate large-scale multi-dimensional data in memory. In this project, we utilized SQL to directly operating on disk-based data which can not fit in memory. To be more specific, we implemented D-Cube\cite{shin2017d} algorithm to detect dense sub-tensors which enables us to discover patterns and anomalies in multi-dimensional data.